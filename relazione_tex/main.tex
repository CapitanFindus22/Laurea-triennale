\documentclass{article}
\usepackage{graphicx}
\usepackage{svg}
\usepackage{placeins}
\usepackage{listings}
\usepackage{enumitem}
\usepackage{float}
\usepackage{algpseudocode}
\usepackage{xcolor}
\usepackage[italian]{babel}

\title{Anomaly detection}
\author{Leonardo Ganzaroli 1961846}
\date{Ottobre 2024}

\begin{document}

\maketitle

\tableofcontents

\newpage

\section{Descrizione generale}
\fontsize{12pt}{14pt}\selectfont
\raggedright

\textbf{Obiettivo:} Il sistema da sviluppare ha lo scopo di rilevare anomalie in stream di dati. \vspace{3pt}
\newline
\textbf{Esecuzione:} Il rilevamento avviene basandosi su:
\vspace{3pt}
\begin{itemize}
    \item Media di ogni stream
    \item Covarianze tra i diversi stream
\end{itemize}
\vspace{3pt}
Un'ulteriore richiesta è che i calcoli vengano effettuati su una finestra temporale.
\vspace{3pt}
\newline
\textbf{Funzioni aggiuntive:} Deve generare degli avvisi in caso di valori ritenuti anomali, ovvero valori che differiscono in modo considerevole dai precedenti. 
\newline

\textbf{Schema generale:}

\vspace{10pt}

\begin{figure}[ht]
	\centering
	\includesvg[width=\textwidth]{Ambiente.svg}
	\caption{Schema generale del sistema}
	\label{fig:schema.svg}
\end{figure}

\newpage

\section{Requisiti Utente}

Lista dei requisiti utente:
\vspace{12pt}

\subsection{Funzionali}

\begin{enumerate}
	\item \textbf{Scelta dei valori di soglia}:
	      \begin{enumerate}[label*=\arabic*.]
	      	\item L'utente deve poter configurare i valori di soglia per la rilevazione delle anomalie del valor medio.
	      	              
	      	\item L'utente deve poter configurare i valori di soglia per la rilevazione delle anomalie della covarianza.
	      \end{enumerate}
	          
	\item \textbf{Finestra temporale}: 
	      \begin{enumerate}[label*=\arabic*.]
	      	\item L'utente deve avere la possibilità di scegliere l'ampiezza della finestra usata per i calcoli.
	      \end{enumerate} 
	      
	\item \textbf{Avvisi anomalie}: 
	      \begin{enumerate}[label*=\arabic*.]
	      	\item L'utente deve poter leggere gli avvisi generati dal sistema riguardo le anomalie.    
	      \end{enumerate}
	          
\end{enumerate}

Si possono visualizzare questi requisiti sottoforma di Use Case nel diagramma di seguito.

\newpage

\textbf{Diagramma Use Case:}
\vspace{10pt}

\begin{figure}[ht]
	\centering
	\includesvg[width=\textwidth]{Use_case.svg}
	\caption{Use Case}
	\label{fig:use_case.svg}
\end{figure}

\clearpage

\section{Requisiti di Sistema}

Lista dei requisiti di sistema:
\vspace{4pt}

\subsection{Funzionali}

\begin{enumerate}
	
	\item \textbf{Lettura dati}:
	      \begin{enumerate}[label*=\arabic*.]
	      	\item Il sistema deve essere in grado di gestire l'arrivo dei dati in tempo reale in base alla capacità della finestra.
	      \end{enumerate}
	      
	\item \textbf{Calcoli statistici}:
	      \begin{enumerate}[label*=\arabic*.]
	      	\item Il sistema deve calcolare il valor medio della finestra associata ad ogni singolo stream e salvarlo sul DB.
	      	              
	      	\item Il sistema deve calcolare le covarianze tra le finestre associate ai diversi stream e salvarle sul DB.
	      \end{enumerate}

	\item \textbf{Determinazione anomalie}:
	      \begin{enumerate}[label*=\arabic*.]
	      	\item Il sistema deve rilevare anomalie nei dati basandosi sulle soglie impostate dall'utente.
	      \end{enumerate}

	\item \textbf{Generazione avvisi}:
	      \begin{enumerate}[label*=\arabic*.]
	      	\item Quando il sistema rileva un'anomalia deve generare un log, stamparlo a schermo e salvarlo sul DB.
	      \end{enumerate}

	\item \textbf{Archiviazione dati}:
	      \begin{enumerate}[label*=\arabic*.]
	      	\item Il sistema deve salvare i valori su un database \textit{PostgreSQL}.
	      \end{enumerate}

	\item \textbf{Input utente}:
	      \begin{enumerate}[label*=\arabic*.]
	      	\item Il sistema deve ricevere e gestire gli input definiti nei requisiti utente.
	      \end{enumerate}    
	          
\end{enumerate}

\subsection{Non Funzionali}

\begin{enumerate}
	\item \textbf{Tempo di invio degli avvisi}:
	      \begin{enumerate}[label*=\arabic*.]
	      	\item L'invio dell'avviso riguardo il valor medio deve avvenire entro mezzo secondo dalla rilevazione dell'anomalia.
	      	      
	      	\item L'invio dell'avviso riguardo la covarianza deve avvenire entro un secondo dalla rilevazione dell'anomalia.
	      \end{enumerate}
	          
\end{enumerate}

\newpage

\section{Struttura}

Si illustra adesso la struttura del sistema più nello specifico.

\subsection{Componenti}

I componenti del sistema sono 4:
\begin{itemize}
	\item Test Generator
	\item Calcolatore media
	\item Calcolatore covarianza
	\item Monitor
\end{itemize}

Segue una breve descrizione.

\subsubsection{Test Generator}
\textbf{Cosa fa:} Invia i dati sugli stream.
\vspace{3pt}
\newline
\textbf{Come lo fa:} Legge una riga per volta da un file CSV ed invia i dati ai rispettivi stream.

\vspace{10pt}
\textbf{Esempio:}
\vspace{5pt} \newline
Se il file ha questo formato:

\vspace{10pt}

\begin{table}[H]
	\centering
	\begin{tabular}{c|c|c|c}
		Colonna 1 & Colonna 2 & Colonna 3 & Colonna 4 \\
		\hline
		12        & 2         & 0         & 1         \\
		32        & 3         & 88        & 111       \\
	\end{tabular}
	\caption{Esempio file}
	\label{tab:esempio}
\end{table}

\vspace{10pt}

La prima riga letta sarà inviata sugli stream in questo modo: 
\vspace{3pt}
\begin{itemize}
    \item STREAM1 $\gets$ 12
    \item STREAM2 $\gets$ 2
    \item STREAM3 $\gets$ 0
    \item STREAM4 $\gets$ 1
\end{itemize}

\subsubsection{Calcolatore media}
\textbf{Cosa fa:} Calcola la media di ogni stream.
\vspace{3pt}
\newline
\textbf{Come lo fa:} Gestisce i valori in entrata tramite una coda (finestra = coda), ogni stream ha la sua coda.
\vspace{3pt}
\newline
\textbf{Altre funzioni:} Salva tutte le medie calcolate nel DB, invia gli elementi di ogni coda e la relativa media al Calcolatore della covarianza.

\subsubsection{Calcolatore covarianza}
\textbf{Cosa fa:} Calcola la covarianza tra le coppie di stream.
\vspace{3pt}
\newline
\textbf{Come lo fa:} Usa le finestre e le medie ricevute dal Calcolatore della media.
\vspace{3pt}
\newline
\textbf{Altre funzioni:} Salva tutte le covarianze calcolate nel DB.

\subsubsection{Monitor}
\textbf{Cosa fanno:} Effettuano dei controlli sul sistema, controllano che i requisiti richiesti vengano rispettati.
\vspace{5pt} 
\newline
In questo caso ne sono stati implementati 5:
\begin{itemize}
	\item Corretto inserimento della media
	\item Corretto inserimento degli avvisi riguardanti la media
	\item Corretto inserimento della covarianza
	\item Tempo trascorso dal riscontro dell'anomalia all'invio dell'avviso (media)
    \item Tempo trascorso dal riscontro dell'anomalia all'invio dell'avviso (covarianza)
\end{itemize}

\newpage

\subsection{Schemi}

I singoli componenti comunicano e collaborano per svolgere i compiti richiesti, a seguire degli schemi che illustrano l'architettura ed il comportamento del sistema.

\vspace{10pt}

\textbf{Singoli componenti e loro collegamenti:}

\vspace{10pt}

\begin{figure}[ht]
	\centering
	\includesvg[width=\textwidth]{Architettura.svg}
	\caption{Componenti del sistema}
	\label{fig:componenti.svg}
\end{figure}

\vspace{12pt}

L'utente fornisce (come visto negli Use Case) le soglie per gli avvisi e la grandezza della finestra, inoltre ha
la possibilità di scegliere il file CSV utilizzato.

\newpage

Anche se i componenti sono indipendenti, è presente una certa linearità nelle operazioni, come si può notare dal seguente diagramma delle attività:

\vspace{10pt}

\begin{figure}[H]
	\centering
	\includesvg[width=\textwidth]{Activity.svg}
	\caption{Activity diagram}
	\label{fig:activity.svg}
\end{figure}

\newpage

Il funzionamento dei monitor si può modellare come una macchina a stati,
per esempio il monitor che controlla il corretto inserimento della media si 
può rappresentare come segue:

\vspace{12pt}

\begin{figure}[ht]
	\includesvg[width=\textwidth]{State_diagram.svg}
	\caption{State Diagram Monitor}
	\label{fig:state_diagram_monitor.svg}
\end{figure}

\vspace{12pt}

I diagrammi degli altri monitor risultano pressoché identici.

\newpage

Ad operazioni già avviate (finestre riempite e tutti gli input dell'utente ricevuti ed elaborati) lo scambio di messaggi tra i vari componenti si può riassumere in questo 
modo:

\vspace{12pt}

\begin{figure}[ht]
	\centering
	\includesvg[width=\textwidth]{MSC.svg}
	\caption{Message Sequence Chart}
	\label{fig:msc.svg}
\end{figure}

\newpage

\section{Implementazione}

Questa sezione contiene le implementazioni dei componenti in pseudocodice e lo schema del database utilizzato.

\subsection{Componenti}

\subsubsection{Test generator}

\begin{algorithmic}
	\State $nome file \gets $INPUT UTENTE
	\State $file \gets apri($nome file$)$
	\State Genera stream pari a $file.colonne$
	\State Invia messaggio con numero di stream
	\While{file ha nuova riga da leggere}
	\State Leggi nuova riga
	\State Invia ogni valore della riga al proprio stream
	\EndWhile
\end{algorithmic}

\subsubsection{Calcolatore media}
\begin{algorithmic}
	\State $dimfinestra \gets $ INPUT UTENTE
	\State $soglia \gets$ INPUT UTENTE
	\State Leggi messaggio
	\State $finestre \gets $ Array di code[valore messaggio]
	\While{Messaggi da leggere disponibili}
	\State Riempi ogni coda fino a $dimfinestra$ da rispettivo stream
	\State Calcola media di ogni stream
    \For{media calcolata in questa iterazione}
            \State Salva media su DB
	      \State Chiama monitor
	      \State Aspetta risposta
       	\If{media diversa da precedente almeno di $soglia$}
            	\State Invia avviso
            	\State Chiama monitor normale e temporale
            	\State Aspetta entrambe le risposte
        	\EndIf
    \EndFor
	\State Invia ogni finestra + sua media a Calcolatore covarianza
	\State Rimuovi da ogni coda elemento più vecchio
	\EndWhile
\end{algorithmic}
\vspace{4pt}

\subsubsection{Calcolatore covarianza}

\begin{algorithmic}
	\State $soglia \gets$ INPUT UTENTE
	\State Leggi messaggio
	\While{Messaggi da leggere disponibili}
	\State Prendi primo messaggio non letto per ogni stream
	\State Calcola covarianza tra tutti i valori dei messaggi (a coppie)
    \For{covarianza calcolata in questa iterazione}
	   \State Salva covarianza su DB
	   \State Chiama monitor
	   \State Aspetta risposta
	   \If{covarianza diversa da precedente almeno di $soglia$}
	       \State Invia avviso
	       \State Chiama monitor normale e temporale
	       \State Aspetta entrambe le risposte
	   \EndIf
    \EndFor
	\EndWhile
\end{algorithmic}

\subsubsection{Monitor}

\begin{algorithmic}
	\While{1}
	\State Aspetta messaggio
    \State Leggi messaggio
	\State Prendi da DB ultimo valore per quello stream
	\State Confronta risultato e messaggio
	\State Salva su DB l'esito del controllo
    \State Invia messaggio
	\EndWhile
\end{algorithmic}

\vspace{1pt}

La struttura è identica per i monitor dello stesso tipo. 
\newline
La struttura dei monitor temporali è invece:

\vspace{1pt}

\begin{algorithmic}
	\While{1}
	\State Aspetta messaggio
    \State Leggi messaggio
	\State Prendi da DB tempo d'inserimento valore per quello stream
    \State Prendi da DB tempo d'inserimento avviso per quello stream
	\State Calcola tempo trascorso e controlla se è nei limiti
	\State Salva su DB tempo + esito del controllo
    \State Invia messaggio
	\EndWhile
\end{algorithmic}

\newpage

\subsection{Database}
Il database contiene 5 tabelle:
\begin{itemize}
	\item \textbf{Session\_info:} mantiene informazioni sulle singole sessioni 
	\item \textbf{Media e Covarianza:} contengono i rispettivi valori
	\item \textbf{Alerts:} contiene le anomalie riscontrate
	\item \textbf{Log\_monitor:} contiene gli esiti dei controlli effettuati dai monitor
\end{itemize}

\vspace{2pt}
Le chiavi delle tabelle sono rappresentate in rosso.
\vspace{2pt}

\begin{table}[H]
    \centering
    \begin{tabular}{|c|c|c|c|}
        \hline
        \textcolor{red}{ID} & Data\_Inizio & Nome\_file & Numero\_colonne\\
        \hline
    \end{tabular}
    \caption{Session\_info}
    \label{tab:session_info}
\end{table}

\vspace{2pt}
Tutti i campi S\_id delle prossime tabelle si riferiscono a ID di Session\_info.
\vspace{2pt}

\begin{table}[H]
    \centering
    \begin{tabular}{|c|c|c|c|}
        \hline
        \textcolor{red}{S\_id} & \textcolor{red}{Data\_ora} & \textcolor{red}{Nome\_stream} & Valore\\
        \hline
    \end{tabular}
    \caption{Media}
    \label{tab:media}
\end{table}

\vspace{2pt}

\begin{table}[H]
    \centering
    \begin{tabular}{|c|c|c|c|c|}
        \hline
         \textcolor{red}{S\_id} & \textcolor{red}{Data\_ora} & \textcolor{red}{Nome\_stream1} & \textcolor{red}{Nome\_stream2} & Valore\\
        \hline
    \end{tabular}
    \caption{Covarianza}
    \label{tab:covarianza}
\end{table}

\vspace{2pt}

\begin{table}[H]
    \centering
    \begin{tabular}{|c|c|c|c|c|c|}
        \hline
        \textcolor{red}{ID} & S\_id & Data\_evento & Nome\_stream & Tipo\_anomalia & Valore\\
        \hline
    \end{tabular}
    \caption{Alerts}
    \label{tab:alerts}
\end{table}

\vspace{2pt}

\begin{table}[H]
    \centering
    \resizebox{\textwidth}{!}{\begin{tabular}{|c|c|c|c|c|c|c|}
        \hline
        \textcolor{red}{ID} & S\_id & Data\_controllo & Nome\_stream & Tipo\_controllo & Esito & Tempo\_trascorso\\
        \hline
    \end{tabular}}
    \caption{Log\_monitor}
    \label{tab:log_monitor}
\end{table}

\newpage

\subsection{Stream Utilizzati}
Per la comunicazione sono stati usati diversi stream di \textit{Redis}.
\vspace{4pt} \newline
Una lista esaustiva:
\begin{itemize}
	\item "INFOSTREAM" usato per comunicare informazioni di servizio
	\item "STREAM*" con $* = [0,1,...,numero\_colonne\_file - 1]$ per inviare i valori dal
	      Test generator al Calcolatore della media
	\item "STREAM\_*" con $* = [0,1,...,numero\_colonne\_file - 1]$ per inviare i valori dal
	      Calcolatore della media al Calcolatore della covarianza
	\item "TMonitor", "AMonitor", "CMonitor", "CTMonitor", "MMonitor" 
	      usati per inviare messaggi ai rispettivi monitor
	\item "M*" con $* = [1,2,3,4,5]$ usati dai monitor per inviare risposte 
\end{itemize}

\section{Risultati Sperimentali}
Il sistema è stato testato con diversi file CSV, per ogni sessione i risultati ottenuti
sono stati in linea con le aspettative meno qualche approssimazione. Inoltre il numero di elementi
inseriti nel DB è stato esattamente quello atteso sia come numero di valori che come numero 
di log generati.

\end{document}